\documentclass[oneside, 12pt]{book}
\usepackage{icdthesisUTF}
\usepackage{tabularx} 
\usepackage{epsfig}
%Τα παρακάτω είναι υποχρεωτικά:

\renewcommand{\thesistitle}{Πρότυπο πτυχιακής εργασίας σε \LaTeX }
\renewcommand{\thesisauthor}{Νικολάου Πεταλίδη (400)}
\renewcommand{\thesisauthorabbrv}{Ν. Πεταλίδης}
\renewcommand{\thesisauthorinitials}{ΝΠ}
\renewcommand{\thesissupervisor}{Δρ. Ε. Καθηγητής, Επιστημονικός Συνεργάτης}
\renewcommand{\thesismonth}{Απρίλιος}
\renewcommand{\thesisyear}{2011}

% Μόνο αν η συγγραφέας είναι γυναίκα 
%\renewcommand{\thesisauthorsex}{female} %if author is female

%Μόνο αν οι συγγραφείς είναι δύο:
\renewcommand{\thesisSecondAuthor}{Κάποιου άλλου (200)}
\renewcommand{\thesisSecondAuthorabbrv}{K. άλλος}
\renewcommand{\thesisSecondAuthorInitials}{ΚΑ}
% Η βιβλιογραφία
\addbibresource{testUTF.bib}

\begin{document}
	% Υποχρεωτικά τα παρακάτω:
	\Titlepage
	\Declarationpage
	\begin{Abstract}
		Εδώ πρέπει να μπει μια περίληψη της πτυχιακής σας. Υπολογίστε 100--150 λέξεις.
	\end{Abstract}
	\tableofcontents
	
	%Μόνο εφόσον θέλετε χωριστό πίνακα για εικόνες και πίνακες
	\listoftables
	\listoffigures
	
	%Προαιρετικά
	\begin{Preface}
		Εδώ μπορεί να μπει πρόλογος. (Δεν είναι απαραίτητο).
	\end{Preface}
	
	%Προαιρετικά
	\begin{Acknowledgement}
		Ευχαριστίες (στο μπαμπά, στη μαμά, κτλ)
	\end{Acknowledgement}
	
	%Προαιρετικά
	\begin{Definitions}
		Ορισμοί εννοιών που μπορεί να είναι χρήσιμοι. Για παράδειγμα:
		
		\begin{description}
			\item [\LaTeX] Σύστημα στοιχειοθεσίας κειμένων
		\end{description}
		
	\end{Definitions}
	
	%Από εδώ αρχίζει το κείμενό σας
	\chapter{Εισαγωγή}
	\leftmark\rightmark
	\section{Η τυπογραφία σήμερα}
	Αυτή είναι η αναφορά σε ένα άρθρο περιοδικού:\citep{Schmidt98}.Αυτή
	είναι η αναφορά σε ένα βιβλίο:\citep{goosens93}. Αυτή είναι η αναφορά
	σε ένα ελληνικό βιβλίο:\citep{Chatzigeorgiou05}. Βιβλίο στα ελληνικά
	με ξένο συγγραφέα:\citep{Sommerville09}. Άρθρο σε
	συνέδριο~\citep{4343930}. 
	
	Τέλος αναφορά σε ιστοσελίδα:~\citep{Wikipedia_BibTeX}.
	
	Εδώ αναφερόμαστε στo σχήμα~\ref{fig:image1}:
	\begin{figure}[h]
		\centering
		\includegraphics[width=35mm]{lion.png}
		\caption{Παράδειγμα εικόνας}
		\label{fig:image1}
	\end{figure}
	
	και εδώ στον πίνακα~\ref{tab:table1}:
	\begin{table}[h]
		\centering
		\caption{Παράδειγμα πίνακα}
		\begin{tabularx}{\linewidth}[h]{|XXX|}%
			\hline
			\hline
			Κίνητρα & Παραδείγματα ευρημάτων & Αριθμός μελετών\\
			\hline
			Ταύτιση με το έργο & Ξεκάθαροι στόχοι &20\\
			Καλό management & Ομαδικότητα &16\\
			Συμμετοχή υπαλλήλων & Συμμετοχή στις αποφάσεις&16\\
			Προοπτικές εξέλιξης & Προοπτικές προαγωγής&15\\
			Ποικιλία στην εργασία & Καλή χρήση ικανοτήτων& 14\\
			Αίσθηση του να ανήκεις κάπου& Υποστηρικτικές σχέσεις&14\\
			Αμοιβές και κίνητρα & Αυξημένος μισθός& 14\\
			\hline
			\hline
		\end{tabularx}
		\label{tab:table1}
	\end{table}
	\appendix
	\chapter{Συνοπτικός οδηγός χρήσης \LaTeX}
	Εδώ βάζετε ότι θα έμπαινε σε παράρτημα.
	\texttt{Δοκιμή σε mono-space}
	%Προαιρετικά
	\begin{Glossary}
		Το γλωσσάρι μπορεί να είναι χρήσιμο αν χρησιμοποιείτε πολλά ακρώνυμα
		και συντομογραφίες. Για παράδειγμα
		\begin{description}
			\item[TCP]Transmission Control Protocol
		\end{description}
	\end{Glossary}
	
	\printbibliography
	\lastpageinfo
\end{document}
