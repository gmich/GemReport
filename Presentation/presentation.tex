%Packages

\documentclass{beamer}
   
\usepackage{graphics}

%------------------------------

%Commands-------------------------

\makeatletter
\def\beamer@calltheme#1#2#3{%
	\def\beamer@themelist{#2}
	\@for\beamer@themename:=\beamer@themelist\do
	{\usepackage[{#1}]{\beamer@themelocation/#3\beamer@themename}}}

\def\usefolder#1{
	\def\beamer@themelocation{#1}
}
\def\beamer@themelocation{}

%------------------------------

%Styles-------------------------

\usefolder{Theme}
\usetheme[progressbar=frametitle]{metropolis}

%------------------------------

%------------------------------
%Cover

\title{Ανάπτυξη εργαλείου για την σχεδίαση παιχνιδιών με την βοήθεια υπολογιστή}
\subtitle{Παρουσίαση πτυχιακής εργασίας}
\date{Μάιος, 2016}
\author[Your Name]{Γιώργος Μιχαηλίδης (3122)
	\\{\small Επιβλέπων: Δρ. Νικόλαος Πεταλίδης}}
\institute{ΤΕΧΝΟΛΟΓΙΚΟ ΕΚΠΑΙΔΕΥΤΙΚΟ ΙΔΡΥΜΑ ΚΕΝΤΡΙΚΗΣ ΜΑΚΕΔΟΝΙΑΣ\\
	ΣΧΟΛΗ ΤΕΧΝΟΛΟΓΙΚΩΝ ΕΦΑΡΜΟΓΩΝ\\
	ΤΜΗΜΑ ΜΗΧΑΝΙΚΩΝ ΠΛΗΡΟΦΟΡΙΚΗΣ ΤΕ}
\titlegraphic{\hfill\includegraphics[height=1.5cm]{Images/logo_teiser}}

%------------------------------

%------------------------------
%Slides

\begin{document}
	\maketitle
	\begin{frame}{Περιεχόμενα}
		\setbeamertemplate{section in toc}[sections numbered]
		\tableofcontents%[hideallsubsections]
	\end{frame}
	
	\section{Εισαγωγή}
		\begin{frame}{Η ανάγκη δημιουργίας ενός εργαλείου για ανάπτυξη λογισμικού}
			Η διαδικασία ανάπτυξης λογισμικού είναι ακριβή και ο σχεδιασμός γίνεται όλο και πιο σύνθετος και περίπλοκος. Τα έργα γίνονται όλο και πιο απαιτητικά και δαπανηρά. Δημιουργήθηκε η ανάγκη για ένα εργαλείο το οποίο να παρέχει ένα ομοιογενές περιβάλλον για την ανάπτυξη σύνθετων έργων. 
			Ένα \alert{CASE (Computer Aided Software Engineering) tool} είναι ένα λογισμικό-εργαλείο το οποίο απλοποιεί τον κύκλο ανάπτυξης ενός λογισμικού. 		
		\end{frame}		
		\begin{frame}{Κοινές λειτουργίες των Case tools}
			\begin{itemize}
				\item Δημιουργία ροής δεδομένων και μοντέλων οντοτήτων.
				\item Καθιέρωση της σχέσης μεταξύ απαιτήσεων και προτύπων.
				\item Ανάπτυξη του σχεδιασμού σε υψηλό επίπεδο.
				\item Ανάπτυξη λειτουργικών και διαδικαστικών περιγραφών
				\item Ανάπτυξη περιπτώσεων δοκιμών (test cases).	
			\end{itemize}		
		\end{frame}	
		\begin{frame}{Σκοπός της πτυχιακής}
			Σκοπός της πτυχιακής είναι να εξηγήσει σε υψηλό επίπεδο τη σύνδεση και αλληλεπίδραση βασικών υποσυστημάτων τα οποία απαρτίζουν μια
			τυπική μηχανή γραφικών, μαζί με διαγράμματα και παραδείγματα χρήσης του API.
		\end{frame}	
		\begin{frame}{Υλοποίηση}
		Η υλοποίηση χωρίζεται σε δύο μεγάλα μέρη.	
		\begin{itemize}
			\item Τη βιβλιοθήκη \alert{Gem Engine} η οποία περιέχει τις κύριες λειτουργίες και τα υποσυστήματα τα οποία απαρτίζουν τη μηχανή γραφικών όπως είναι, το σύστημα εισόδων και η διαχείριση οθονών.
			\item Tο γραφικό περιβάλλον της μηχανής \alert{Gem IDE} το οποίο προσφέρει οπτική αναπαράσταση των λειτουργιών του \alert{Gem Engine}. Μέσω του \alert{Gem IDE} o χρήστης μπορεί να αποσφαλματώσει το παιχνίδι και να προσθέσει λειτουργίες μέσω γραφικού περιβάλλοντος.
		\end{itemize}
		\end{frame}			
		
	\section{Mηχανές Γραφικών}	
	\begin{frame}{Τι είναι μια μηχανή γραφικών}		
		Στο τομέα του σχεδιασμού παιχνιδιών το πιο διαδεδομένο CASE tool είναι η \alert{μηχανή γραφικών}. Μια μηχανή γραφικών είναι μια σουίτα από επαναχρησιμοποιήσιμα οπτικά εργαλεία, τα οποία βρίσκονται σε ένα ενιαίο περιβάλλον. 
	\end{frame}
	
	\begin{frame}{Τι παρέχει μια μηχανή γραφικών}		
	\begin{enumerate}
		\item Φωτοαπόδοση σε πραγματκό χρόνο (real time rendering)
		\item Mηχανή φυσικής και εντοπισμος συγκρούσεων (physics and collision detection)
		\item Scripting
		\item Animation
		\item Τεχνητή νοημοσύνη (artificial intelligence)
		\item Δικτύωση (networking)
		\item Παραλληλισμό ενεργειών (multitasking)
		\item Διαχείριση μνήμης
		\item Γράφο σκηνής (scene graph)
	\end{enumerate}
	\cite{gregory2009game}
	\end{frame}
		
	\begin{frame}{Ανάπτυξη ηλεκτρονικών παιχνιδιών}
		Ανάπτυξη ηλεκτρονικών παιχνιδιών (game development) ονομάζουμε τη διαδικασία της δημιουργίας ενός παιχνιδιού. Η ομάδα ανάπτυξης μπορεί να κυμαίνεται από ένα άτομο μέχρι μια μεγάλη επιχείρηση.	
	\end{frame}
	
	\begin{frame}{Σχεδιασμός παιχνιδιών}
		Σχεδιασμό παιχνιδιών (game design) ονομάζουμε τον σχεδιασμό και την εφαρμογή τεχνικών αισθητικής στη δημιουργία ενός παιχνιδιού με σκοπό τη διευκόλυνση της αλληλεπίδρασης μεταξύ των παικτών. Οι μηχανές γραφικών χρησιμοποιούνται για σκοπούς δημιουργίας παιχνιδιών.
	\end{frame}
	
	\section{Bιβλιοθήκη}
		
	\begin{frame}{Κύκλος ζωής του πυρήνα}
		\begin{figure}
			\centering
			\resizebox{5.0cm}{!}{\input{Images/Lifecycle/lifecycle.pdf_tex}}
		\end{figure}
		\cite{citeulike:13049596}	
	\end{frame}

	\begin{frame}{Φυσική και εντοπισμός συγκρούσεων}
		\begin{figure}
			\centering
			\resizebox{10.5cm}{!}{\input{Images/PhysicsEngine/physics_engine.pdf_tex}}
		\end{figure}
		\cite{realtime_collision04}
	\end{frame}
	
	\begin{frame}{Δέντρα συμπεριφορών}
		\begin{figure}
			\centering
			\resizebox{10.5cm}{!}{\input{Images/AI/behavior_trees.pdf_tex}}
		\end{figure}
		\cite{champandard2007understanding}
	\end{frame}

	\begin{frame}{Δικτύωση}
		Δικτύωση στα ηλεκτρονικά παιχνίδια έχουμε όταν περισσότεροι από ένας παίχτες σε διαφορετικές πλατφόρμες ή υπολογιστές, μοιράζονται και αλληλεπιδρούν στο ίδιο εικονικό περιβάλλον. Παίχτες σε διάφορα σημεία του πλανήτη θέλουν να μοιραστούν ένα εικονικό περιβάλλον σε πραγματικό χρόνο με σκοπό την συνεργασία ή την αντιπαλότητα.
		\cite{sanja14}
	\end{frame}
	
	\begin{frame}{Βρόγχος ενημέρωσης της δικτύωσης}
		\begin{figure}
			\centering
			\resizebox{10.5cm}{!}{\input{Images/Network/network_sequence.pdf_tex}}
		\end{figure}
	\end{frame}
	
	\section{Διαγνωστικά και Αποσφαλμάτωση}
	\begin{frame}{Διαγνωστικά και αποσφαλμάτωση κατά την ανάπτυξη}
		Η ανάπτυξη του λογισμικού πρέπει να παρακολουθείται συνεχώς. Σε λογισμικά όπως ένα παιχνίδι, η επίδοση της εφαρμογής είναι εξαιρετικά σημαντική. Για τη μετάβαση από δοκιμαστικές σε τελικές εκδόσεις, χρειάζεται συλλογή και ανάλυση διαγνωστικών επιδόσεων για διαφορετικές πλατφόρμες, επεξεργαστές και κάρτες γραφικών. Ακόμη και στις τελικές εκδόσεις όμως υπάρχει η πιθανότητα σφάλματος.         \cite{richter2012clr}
	\end{frame}
	
	\begin{frame}{Υποσυστήματα αποσφαλμάτωσης Ι}
		\begin{itemize}
			\item Logging and Tracing
			\item Debug Drawing API για απόδοση γεωμετρίας.
			\item Time Benchmarker το οποίο προσφέρει δυναμική ανάλυση του χρόνου εκτέλεσης και των ενημερώσεων.
			\item Μενού επιλογών στο παιχνίδι με δυνατότητα εναλλαγής επιλογών και τροποποίησης τιμών.
			\item Debug camera η οποία κινείται χωρίς περιορισμούς στο χώρο.
			\item Κονσόλα η οποία εκτελεί scripts και εντολές υπό τη μορφή κειμένου παρόμοια με το Unix Shell.
			\item Δυνατότητα παύσης.
			\item Cheats.		
		\end{itemize}
	\end{frame}			
	
	\begin{frame}{Υποσυστήματα αποσφαλμάτωσης II}
		\begin{itemize}	
			\item Assertions εντολές οι οποίες αν δεν αξιολογηθούν ως αληθές κατά το χρόνο εκτέλεσης του λογισμικού, τερματίζουν το λογισμικό με κωδικό σφάλματος						
			\item Screenshots και screen captures.
			\item Ιngame profiler και profiling blocks με αναγνώσιμα ονόματα.
			\item Ιεραρχικό Profiling.
			\item Εξαγωγή δεδομένων σε μορφή η οποία είναι εύκολα αναγνώσιμη από άνθρωπο για αποσφαλμάτωση.
			\item Στατιστικά μνήμης και ανίχνευση διαρροών.
			\item FPS Counter το οποίο αποδίδει στην οθόνη τον αριθμό τον FPS.
		\end{itemize}
	\end{frame}
	
	\section{Συμπεράσματα}
	\begin{frame}{Περιορισμοί}
		Με ένα εργαλείο όπως μια μηχανή γραφικών, ο σχεδιαστής παιχνιδιών απαλλάσσεται από το βάρος της σχεδίασης, τις τεχνικές λεπτομέρειες των πλατφορμών και εστιάζει καθαρά στην ιδέα του. Λόγω της γενικότητας όμως του εργαλείου, είναι δύσκολο να αντιμετωπίσει όλα τα προβλήματα της ανάπτυξης παιχνιδιών.
	\end{frame}		
	\begin{frame}{Επέκταση}
		Η μηχανή προσφέρει δυνατότητες επεκτασιμότητας μέσω plugins. \cite{Erl:2009:SDP:1538586}
		Επέκταση του \alert{Gem IDE} μπορεί να γίνει χρησιμοποιώντας το \alert{Gem IDE Core} και με εξαγωγή τη βιβλιοθήκης σαν \alert{Module}.
	\end{frame}		
	
	\plain{Ερωτήσεις;}
	\plain{Ευχαριστώ για το χρόνο σας}	
	
	\begin{frame}[allowframebreaks]{Βιβλιογραφία}
		
		\bibliography{../Content/bibliography} 
		\bibliographystyle{abbrv}
		
	\end{frame}
	
\end{document}

