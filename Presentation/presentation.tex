%Packages

\documentclass{beamer}
   
\usepackage{graphics}
\usepackage[backend=bibtex]{biblatex}

%------------------------------

%Commands-------------------------

\makeatletter
\def\beamer@calltheme#1#2#3{%
	\def\beamer@themelist{#2}
	\@for\beamer@themename:=\beamer@themelist\do
	{\usepackage[{#1}]{\beamer@themelocation/#3\beamer@themename}}}

\def\usefolder#1{
	\def\beamer@themelocation{#1}
}
\def\beamer@themelocation{}

\newcommand\blfootnote[1]{%
	\begingroup
	\renewcommand\thefootnote{}\footnote{#1}%
	\addtocounter{footnote}{-1}%
	\endgroup
}

%------------------------------

%Styles-------------------------

\usefolder{Theme}
\usetheme[progressbar=frametitle]{metropolis}

%------------------------------

%------------------------------
%Cover

\title{Ανάπτυξη εργαλείου για την σχεδίαση παιχνιδιών με την βοήθεια υπολογιστή}
\subtitle{Παρουσίαση πτυχιακής εργασίας}
\date{Μάιος, 2016}
\author[Your Name]{Γιώργος Μιχαηλίδης (3122)
	\\{\small Επιβλέπων: Δρ. Νικόλαος Πεταλίδης}}
\institute{ΤΕΧΝΟΛΟΓΙΚΟ ΕΚΠΑΙΔΕΥΤΙΚΟ ΙΔΡΥΜΑ ΚΕΝΤΡΙΚΗΣ ΜΑΚΕΔΟΝΙΑΣ\\
	ΣΧΟΛΗ ΤΕΧΝΟΛΟΓΙΚΩΝ ΕΦΑΡΜΟΓΩΝ\\
	ΤΜΗΜΑ ΜΗΧΑΝΙΚΩΝ ΠΛΗΡΟΦΟΡΙΚΗΣ ΤΕ}
\titlegraphic{\hfill\includegraphics[height=1.5cm]{Images/logo_teiser}}

%------------------------------

%------------------------------
%Slides

\addbibresource{Content/bibliography.bib}

\begin{document}
	\maketitle
	%\begin{frame}{Περιεχόμενα}
	%	\setbeamertemplate{section in toc}[sections numbered]
	%	\tableofcontents%[hideallsubsections]
	%\end{frame}
	
	\section{Εισαγωγή}
	
	\begin{frame}{Η ανάγκη δημιουργίας ενός εργαλείου για ανάπτυξη λογισμικού}
			\begin{itemize}
			\item Η ανάπτυξη λογισμικού είναι ακριβή, περίπλοκη και σύνθετη.
			\item Τα έργα είναι απαιτητικά και δαπανηρά.
			\end{itemize}
			
			\alert{CASE (Computer Aided Software Engineering) tool}: ένα λογισμικό-εργαλείο το 
			οποίο απλοποιεί τον κύκλο ανάπτυξης ενός λογισμικού και παρέχει ένα ομοιογενές περιβάλλον για την ανάπτυξη σύνθετων έργων. \footfullcite{oracle:case}

	\end{frame}		
		
	\begin{frame}{Κοινές λειτουργίες των Case tools}
			\begin{enumerate}
				\item Δημιουργία ροής δεδομένων και μοντέλων οντοτήτων.
				\item Καθιέρωση της σχέσης μεταξύ απαιτήσεων και προτύπων.
				\item Ανάπτυξη του σχεδιασμού σε υψηλό επίπεδο.
				\item Ανάπτυξη λειτουργικών και διαδικαστικών περιγραφών
				\item Ανάπτυξη περιπτώσεων δοκιμών (test cases).	
			\end{enumerate}		
	\end{frame}
		
	\begin{frame}{Σκοπός της πτυχιακής}
		 Eπεξήγηση σε υψηλό επίπεδο της υλοποίησης, σύνδεσης και αλληλεπίδρασης των βασικών υποσυστημάτων μιας τυπικής μηχανής γραφικών με διαγράμματα UML, ροής, ακολουθίας, χρήσης και παραδείγματα του API.		
	\end{frame}	
		
	\begin{frame}{Υλοποίηση}
			Βασίζεται σε SOLID principles. \footfullcite{hall2014adaptive}
			\newline
			Ακολουθεί functional και αντικειμενοστραφείς τεχνικές.
			\newline 
			\newline
			\alert{Gem Engine:} Οι κύριες λειτουργίες και τα υποσυστήματα τα οποία απαρτίζουν τη μηχανή γραφικών.
			\newline
			\alert{Gem IDE:} Οπτική αναπαράσταση των λειτουργιών του \textit{Gem Engine}.
	\end{frame}			
		
	\begin{frame}{Πως ορίζεται ένα παιχνίδι; I}
		\begin{itemize}
			\item Ένα σύνολο παραγόντων οι οποίοι δρουν βάσει στρατηγικών και τεχνικών, με στόχο να μεγιστοποιήσουν τα κέρδη μέσα στον εικονικό κόσμο, μέσα σε ένα πλαίσιο καλά ορισμένων κανόνων. \footfullcite{gametheory}
			\item Mια διαδραστική εμπειρία, η οποία παρέχει στον χρήστη μια σειρά
			από πρότυπα (patterns) τα οποία μαθαίνει και στην τελική εξειδικεύεται.
			\footfullcite{koster04}
		\end{itemize}
	\end{frame}
	
	\begin{frame}{Πως ορίζεται ένα παιχνίδι; II}
		\begin{itemize}
			\item Η εθελοντική προσπάθεια να ξεπεραστούν περιττά εμπόδια.
			\footfullcite{suits2005grasshopper}
			\item Ένα κλειστό σύστημα που δεσμεύει τους παίκτες σε ένα δομημένο περιβάλλον σύγκρουσης και επιλύει την αβεβαιότητά του με ένα άνισο αποτέλεσμα.
			\footfullcite{fullerton2008game}
		\end{itemize}
	\end{frame}
	
	\begin{frame}{Mηχανή Γραφικών}		
		Mια σουίτα από επαναχρησιμοποιήσιμα οπτικά εργαλεία, τα οποία βρίσκονται σε ένα ενιαίο περιβάλλον. 
		\newline
		\newline
		\alert{Προγραμματισμός (programming)}: η λήψη των προδιαγραφών σχεδιασμού των σχεδιαστών.
		\newline
		\alert{Δημιουργία (design)}: ο σχεδιασμός και εφαρμογή τεχνικών αισθητικής.		
	\end{frame}
	
	\begin{frame}{Τι παρέχει μια μηχανή γραφικών}		
	\begin{enumerate}
		\item Φωτοαπόδοση σε πραγματκό χρόνο (real time rendering)
		\item Mηχανή φυσικής και εντοπισμός συγκρούσεων
		\item Scripting
		\item Animation
		\item Τεχνητή νοημοσύνη
		\item Δικτύωση (networking)
		\item Παραλληλισμό ενεργειών (multitasking)
		\item Διαχείριση μνήμης
		\item Γράφο σκηνής (scene graph)
	\end{enumerate}
	\blfootnote{\fullcite{gregory2009game}}
	\end{frame}
		
	\section{Bιβλιοθήκη}
		
	\begin{frame}{Κύκλος ζωής του πυρήνα}
		\begin{figure}
			\centering
			\resizebox{4.8cm}{!}{\input{Images/Lifecycle/lifecycle.pdf_tex}}
		\end{figure}
		\blfootnote{\fullcite{citeulike:13049596}}	
	\end{frame}

	\begin{frame}{Φυσική και εντοπισμός συγκρούσεων}
		\begin{figure}
			\centering
			\resizebox{10.5cm}{!}{\input{Images/PhysicsEngine/physics_engine.pdf_tex}}
		\end{figure}
		\blfootnote{\fullcite{realtime_collision04}}
	\end{frame}
	
	\begin{frame}{Δέντρα συμπεριφορών}
		\begin{figure}
			\centering
			\resizebox{10.5cm}{!}{\input{Images/AI/behavior_trees.pdf_tex}}
		\end{figure}
		\blfootnote{\fullcite{champandard2007understanding}}
	\end{frame}

	\begin{frame}{Δικτύωση}
		Περισσότεροι από ένας παίχτες σε διαφορετικές πλατφόρμες ή υπολογιστές, μοιράζονται και αλληλεπιδρούν στο ίδιο εικονικό περιβάλλον.
		
		\alert{Αρχιτεκτονικές δικτύου} \footfullcite{sanja14}		
		\begin{itemize}
			\item Client-server model
			\item Client on top of server model
			\item Peer to peer
		\end{itemize}			
	
	\end{frame}
	
%	\begin{frame}{Βρόγχος ενημέρωσης της δικτύωσης}
%		\begin{figure}
%			\centering
%			\resizebox{10.5cm}{!}{\input{Images/Network/network_sequence.pdf_tex}}
%		\end{figure}
%	\end{frame}
	
	\section{Διαγνωστικά και Αποσφαλμάτωση}
	\begin{frame}{Διαγνωστικά και αποσφαλμάτωση κατά την ανάπτυξη}
		Η ανάπτυξη του λογισμικού πρέπει να παρακολουθείται συνεχώς. 
		\newline 
		\newline 
		Για τη μετάβαση από δοκιμαστικές σε τελικές εκδόσεις, χρειάζεται συλλογή και ανάλυση διαγνωστικών επιδόσεων για διαφορετικές πλατφόρμες, επεξεργαστές και κάρτες γραφικών. \footfullcite{richter2012clr}
	\end{frame}
	
	\begin{frame}{Υποσυστήματα αποσφαλμάτωσης Ι}
		\begin{itemize}
			\item Logging and Tracing
			\item Debug Drawing API για απόδοση γεωμετρίας.
			\item Time Benchmarker το οποίο προσφέρει δυναμική ανάλυση του χρόνου εκτέλεσης και των ενημερώσεων.
			\item Μενού επιλογών στο παιχνίδι με δυνατότητα εναλλαγής επιλογών και τροποποίησης τιμών.
			\item Debug camera η οποία κινείται χωρίς περιορισμούς στο χώρο.
			\item Κονσόλα η οποία εκτελεί scripts και εντολές υπό τη μορφή κειμένου παρόμοια με το Unix Shell.	
		\end{itemize}
	\end{frame}			
	
	\begin{frame}{Υποσυστήματα αποσφαλμάτωσης II}
		\begin{itemize}	
			\item Assertions εντολές οι οποίες αν δεν αξιολογηθούν ως αληθείς κατά το χρόνο εκτέλεσης του λογισμικού, τερματίζουν το λογισμικό με κωδικό σφάλματος.				
			\item Screenshots και screen captures.
			\item Ιngame profiler και profiling blocks με αναγνώσιμα ονόματα.
			\item Εξαγωγή δεδομένων σε μορφή η οποία είναι εύκολα αναγνώσιμη από άνθρωπο για αποσφαλμάτωση.
			\item Στατιστικά μνήμης και ανίχνευση διαρροών.
			\item FPS Counter
		\end{itemize}
	\end{frame}
	
	\section{Συμπεράσματα}
	\begin{frame}{Περιορισμοί}
		O σχεδιαστής παιχνιδιών απαλλάσσεται από το βάρος της σχεδίασης, τις τεχνικές λεπτομέρειες των πλατφορμών και εστιάζει καθαρά στην ιδέα του.
		\newline
		\newline
		Λόγω της γενικότητας του εργαλείου, είναι δύσκολο να αντιμετωπιστούν όλα τα προβλήματα της ανάπτυξης παιχνιδιών.
	\end{frame}		
	\begin{frame}{Επέκταση}
		Η μηχανή προσφέρει δυνατότητες επεκτασιμότητας μέσω plugins. \footfullcite{Erl:2009:SDP:1538586}
		\newline
		\newline
		Επέκταση του \alert{Gem IDE} μπορεί να γίνει χρησιμοποιώντας το \alert{Gem IDE Core} και με εξαγωγή της βιβλιοθήκης ως \alert{Module}.
	\end{frame}		
	
	\plain{Ερωτήσεις;}
	\plain{Ευχαριστώ}	
	
	\begin{frame}[allowframebreaks]{Βιβλιογραφία}
		\printbibliography[heading=none]
	\end{frame}
	
\end{document}

