\chapter{Εισαγωγή}
	\paragraph{Σκοπός} 
	Σκοπός της πτυχιακής είναι να εξηγήσει θεωρητικά και πρακτικά τα κομμάτια που απαρτίζουν μια
	τυπική μηχανή γραφικών και πώς συνδέονται αρχιτεκτονικά μεταξύ τους, μαζί με παραδείγματα και οδηγίες για την επέκτασή τους. Για το κομμάτι του rendering, δεν  έγινε απευθείας με προτόκολο που επικοινωνεί με την κάρτα γραφικών, αλλά με την ανοικτού λογισμικού βιβλιοθήκη monogame η οποία είναι και cross-platform,  δηλαδή ανάλογα με το περιβαλλον ανάπτυξης χρησιμοποιεί το ανάλογο προτόκολο για επικοινωνία με την κάρτα γραφικών, για παράδειγμα για Linux OS χρησιμοποιεί OpenGL και για Android ΟpenGL ES, για να επικεντρωθεί στο rendering γενικά και όχι για συγκεκριμένη πλατφόρμα. Τα παραδείγματα αξιοποιούν object-oriented και functional paradigms και είναι γραμμένα σε C\#. Ο κώδικας βρίσκεται στο \cite{gem} και \cite{ginet} και η τεκμηρίωση στο \cite{gemDocs}.
	
	\paragraph{Θεμελίωση} 	
	Οι μηχανές γραφικών διαφέρουν με βάση τις λεπτομέριες αρχιτεκτονικής και υλοποίησης, αλλά τα πρότυπα σχεδίασης είναι καθολικά. Πρακτικά, οι μηχανές γραφικών απαρτίζονται από τις ίδιες βασικές έννοιες. Πολλά βιβλία έχουν γραφτεί τα οποία εστιάζουν στην κάθε έννοια ξεχωριστά αλλά ελάχιστα για το πως αυτά τα στοιχεία επικοινωνούν και αλληλεπιδρούν μεταξύ τους. 
	Η εργασία εστιάζει στην αρχιτεκτονική των μηχανών, 
	για το πώς οι ομάδες είναι οργανωμένες για να δουλεύουν μεταξύ τους, 
	ποια συστήματα και μοτίβα επαναλαμβάνονται στη δημιουργία μηχανών, 
	ποιες είναι οι απαιτήσεις για το κάθε μεγάλο υποσύστημα της μηχανής, 
	ποια συστήματα είναι αγνωστικιστικά σε παιχνίδια ή σε είδη παιχνιδιών και ποια είναι συγκεκριμένα,
	και πότε σταματά η μηχανή και ξεκινά η υλοποίηση του παιχνιδιού.
	
	Eπιπρόσθετα, θα γίνει σύγκριση με άλλες δημοφιλείς μηχανές γραφικών και βιβλιοθηκών τεχνικές για configuration management, versioning και διάφορα συστήματα ανάπτυξης. 
\section{Δομή της εργασίας}
\begin{enumerate}
	\item{Game Development}
	\item{Ο πηρύνας}
	\item{Δικτύωση}
	\item{Διαγνωστικά}
	\item{Επίλογος}
\end{enumerate}
\section{Εργαλεία και Frameworks}
\begin{itemize}
	\item{Monogame}
	\item{OpenGL}
	\item{TPL}
	\item{Autofac}
	\item{NLog}
	\item{Glsl}
	\item{MEF}
	\item{WPF}
	\item{Roslyn}
	\item{Rx}
	\item{Caliburn}
	\item{Gemini}
\end{itemize}
\section{Τεχνικές υλοποίησης}
Παραδείγματα από unity και unreal.
\section{Case Tools}
Η διαδικασία ανάπτυξης λογισμικού είναι ακριβή και ο σχεδιασμός γίνεται όλο πιο σύνθετος και περίπλοκος. Τα έργα γίνονται όλο πιο απαιτητικά και δαπανηρά. Δημιουργήθηκε η ανάγκη για ένα εργαλείο το οποίο να παρέχει ένα ομοιογενές περιβάλλον για την ανάπτυξη σύνθετων έργων. 
Ένα CASE (Computer Aided Software Engineering) tool είναι ένα λογισμικό-εργαλείο το οποίο απλοποιεί τον κύκλο ανάπτυξης ενός λογισμικού. Τα Case Tools γίνονται όλο και πιο δημοφιλές, λόγω της βελτίωσης των δυνατοτήτων και της λειτουργικότητας στην ανάπτυξη της ποιότητας του λογισμικού. Η διαδικασία ανάπτυξης αυτοματοποιείται, και συντονίζεται. Το λογισμικό συντηρείται και αναλύεται εύκολα. 

\subsection{Kοινές λειτουργίες}
\begin{itemize}
	\item Δημιουργία ροής δεδομένων και μοντέλων οντοτήτων.
	\item Καθιέρωση της σχέσης μεταξύ απαιτήσεων και προτύπων.
	\item Ανάπτυξη του σχεδιασμού σε υψηλό επίπεδο.
	\item Ανάπτυξη λειτουργικών και διαδικαστικών περιγραφών
	\item Ανάπτυξη περιπτώσεων δοκιμών (test cases).	
\end{itemize}

\subsection{Γιατί;}
\begin{itemize}
	\item Γρήγορη εγκατάσταση
	\item Εξοικονόμηση χρόνου μειώνοντας τον χρόνο στον προγραμματισμό και στις δοκιμές.
	\item Οπτικοποίηση του κώδικα και της ροής δεδομένων
	\item Βέλτιστη χρήση των διαθέσιμων πόρων.
	\item Ανάλυση, ανάπτυξη και σχεδιασμό με ενιαίες μεθοδολογίες.
	\item Δημιουργία και τροποποίηση τεκμηρίωσης (documentation)
	\item Αποτελεσματική μεταφορά πληροφοριών ανάμεσα στα διάφορα εργαλεία
	\item Γρήγορη δημιουργία λογισμικού.
\end{itemize}

\subsection{Χρήση}
\begin{itemize}
	\item Για να διευκολυνθεί η μεθοδολογία σχεδιασμού. 
	\item Για Rapid Application Development
	\item Testing
	\item Documentation
	\item Project Management
	\item Μειωμένο κόστος συντήρησης
	\item Αύξηση της παραγωγικότητας:
\end{itemize}
H Αυτοματοποίηση των διαφόρων δραστηριοτήτων των διαδικασιών ανάπτυξης και διαχείρισης του συστήματος αυξάνει την παραγωγικότητα της ομάδας ανάπτυξης.

\subsection{Χαρακτηριστικά ενός καλού case tool}
\begin{itemize}
	\item Τυποποιημένη μεθοδολογία χρησιμοποιώντας τεχνικές μοντελοποίησης όπως UML.
	\item Flexibility: το εργαλείο πρέπει να προσφέρει τη δυνατότητα στο χρήστη να επιλέγει ποια εργαλεία να χρησιμοποιήσει.
	\item Strong integration: το εργαλείο πρέπει να υποστηρίζει όλα τα στάδια ανάπτυξης. Όταν γίνεται μια αλλαγή, τα στάδια τα οποία επιρεάζονται πρέπει να τροποποιούνται κατάλληλα.
	\item Ενσωμάτωση με εργαλεία ελέγχου.
	\item Reverse-engineering: δυνατότητα δημιουργίας κώδικα από δεδομένα
\end{itemize}

	
