\chapter{Εισαγωγή}
	\paragraph{Σκοπός} 
	Σκοπός της πτυχιακής είναι να εξηγήσει σε υψηλό επίπεδο ποια βασικά υποσυστήματα τα οποία απαρτίζουν μια
	τυπική μηχανή γραφικών και πώς συνδέονται αρχιτεκτονικά μεταξύ τους, μαζί με διαγράμματα και παραδείγματα χρήσης του \gls{API}. 
	
	\paragraph{Θεμελίωση} 	
	Οι διαφορές των μηχανών γραφικών διαφέρουν στις λεπτομέριες αρχιτεκτονικής και υλοποίησης, αλλά έχουν πανομοιότυπη δομή. Πρακτικά, οι μηχανές γραφικών απαρτίζονται από τα ίδια υποσυστήματα. Πολλά βιβλία έχουν γραφτεί τα οποία εστιάζουν στο κάθε υποσύστημα ξεχωριστά, όπως π.χ. το rendering.
	Η εργασία εστιάζει στην αρχιτεκτονική των υποσυστημάτων, για το πώς τα υποσυστήματα είναι οργανωμένα, ποια συστήματα και μοτίβα επαναλαμβάνονται στη δημιουργία μηχανών, ποιες είναι οι απαιτήσεις για το κάθε μεγάλο υποσύστημα της μηχανής και πώς οργανώνεται ένα υποσύστημα με χωρίς να έχει δεσμεύσεις με συγκεκριμνένη πλατφόρμα.

	\paragraph{Υλοποίηση}
	Η υλοποίηση χωρίζεται σε δύο μεγάλα μέρη.	
	\begin{itemize}
		\item {Gem Engine} η βιβλιοθήκη η οποία περιέχει τα υποσυστήματα που απαρτίζουν τη μηχανή γραφικών 
		\item {Gem IDE} το γραφικό περιβάλλον της μηχανής και προσφέρει οπτική αναπαράσταση τον λειτουργιών του Gem Engine.
	\end{itemize}
	H υλοποιήση αξιοποιεί object-oriented και functional paradigms και είναι γραμμένα σε C\#. Ο κώδικας βρίσκεται στο \cite{gem} και \cite{ginet} και η τεκμηρίωση στο \cite{gemDocs}.	
	
\section{Δομή της εργασίας}
\begin{enumerate}
	\item{Game Development} Επεξήγηση των εννοιών και διαφορών του game development και game design. Ανάλυση μιας τυπικής ομάδας ανάπτυξης παιχνιδιών, στην οποία απευθύνεται το εργαλείο και οριοθέτηση του τι θεωρείται παιχνίδι.
	\item{Ο πηρύνας της μηχανής} Ανάλυση κάποιων βασικών υποσυστημάτων τα που βρίσκονται στον πηρύνα και είναι αναγκαία για τη λειτουργία της μηχανής.
	\item{Δικτύωση} Περιγραφή του υποσυστήματος δικτύωσης της μηχανής.
	\item{Διαγνωστικά και αποσφαλμάτωση} Τεχνικές ελέγχου υγείας του λογισμικού τόσο κατά τη διαδικασία δημιουργίας όσο και στις τελικές εκδόσεις.
	\item{Επίλογος} Συμπεράσματα και τεχνικές επέκτασης.
\end{enumerate}
\section{Εργαλεία και Frameworks}
Τα βασικά εργαλεία τα οποία χρησιμοποιεί ή επεκτίνει η μηχανή είναι
\begin{itemize}
	\item{Monogame} Cross-platform rendering το οποίο βρίσκεται ένα επίπεδο πάνω από τα drivers της κάρτας γραφικών.
	\item{TPL} Ασύγχρονα πρότυπα για ταυτόχρονο και παράλληλο προγραμματισμό 
	\item{Lidgren} Αφαίρεση για networks sockets. 
	\item{Castle Proxy} Δημιουργία proxy κλάσεων σε επίπεδο IL
	\item{Farseer} Προσομοίωση φυσικής
	\item{Autofac} IOC container
	\item{NLog} Logger
	\item{Glsl} Shaders
	\item{MEF} Διαχείριση Plugin
	\item{WPF} Desktop application framework για το IDE
	\item{Roslyn} C\# compiler ανοιχτού κώδικα με δυνατότητα scripting
	\item{Rx} Observable streams 
	\item{Caliburn} MV-VM για WPF
	\item{Moq} Mocking Library
	\item{MsTest} Unit Testing 
	\item{VS 2015 Ultimate} IDE και profiler
\end{itemize}

\section{Τεχνικές υλοποίησης}
Παραδείγματα από unity και unreal data structures
\section{Case Tools}
Η διαδικασία ανάπτυξης λογισμικού είναι ακριβή και ο σχεδιασμός γίνεται όλο πιο σύνθετος και περίπλοκος. Τα έργα γίνονται όλο πιο απαιτητικά και δαπανηρά. Δημιουργήθηκε η ανάγκη για ένα εργαλείο το οποίο να παρέχει ένα ομοιογενές περιβάλλον για την ανάπτυξη σύνθετων έργων. 
Ένα CASE (Computer Aided Software Engineering) tool είναι ένα λογισμικό-εργαλείο το οποίο απλοποιεί τον κύκλο ανάπτυξης ενός λογισμικού. Τα Case Tools γίνονται όλο και πιο δημοφιλές, λόγω της βελτίωσης των δυνατοτήτων και της λειτουργικότητας στην ανάπτυξη της ποιότητας του λογισμικού. Η διαδικασία ανάπτυξης αυτοματοποιείται, και συντονίζεται. Το λογισμικό συντηρείται και αναλύεται εύκολα. 

\subsection{Kοινές λειτουργίες}
\begin{itemize}
	\item Δημιουργία ροής δεδομένων και μοντέλων οντοτήτων.
	\item Καθιέρωση της σχέσης μεταξύ απαιτήσεων και προτύπων.
	\item Ανάπτυξη του σχεδιασμού σε υψηλό επίπεδο.
	\item Ανάπτυξη λειτουργικών και διαδικαστικών περιγραφών
	\item Ανάπτυξη περιπτώσεων δοκιμών (test cases).	
\end{itemize}

\subsection{Γιατί;}
\begin{itemize}
	\item Γρήγορη εγκατάσταση
	\item Εξοικονόμηση χρόνου μειώνοντας τον χρόνο στον προγραμματισμό και στις δοκιμές.
	\item Οπτικοποίηση του κώδικα και της ροής δεδομένων
	\item Βέλτιστη χρήση των διαθέσιμων πόρων.
	\item Ανάλυση, ανάπτυξη και σχεδιασμό με ενιαίες μεθοδολογίες.
	\item Δημιουργία και τροποποίηση τεκμηρίωσης (documentation)
	\item Αποτελεσματική μεταφορά πληροφοριών ανάμεσα στα διάφορα εργαλεία
	\item Γρήγορη δημιουργία λογισμικού.
\end{itemize}

\subsection{Χρήση}
\begin{itemize}
	\item Για να διευκολυνθεί η μεθοδολογία σχεδιασμού. 
	\item Για Rapid Application Development
	\item Testing
	\item Documentation
	\item Project Management
	\item Μειωμένο κόστος συντήρησης
	\item Αύξηση της παραγωγικότητας:
\end{itemize}
H Αυτοματοποίηση των διαφόρων δραστηριοτήτων των διαδικασιών ανάπτυξης και διαχείρισης του συστήματος αυξάνει την παραγωγικότητα της ομάδας ανάπτυξης.

\subsection{Χαρακτηριστικά ενός καλού case tool}
\begin{itemize}
	\item Τυποποιημένη μεθοδολογία χρησιμοποιώντας τεχνικές μοντελοποίησης όπως UML.
	\item Flexibility: το εργαλείο πρέπει να προσφέρει τη δυνατότητα στο χρήστη να επιλέγει ποια εργαλεία να χρησιμοποιήσει.
	\item Strong integration: το εργαλείο πρέπει να υποστηρίζει όλα τα στάδια ανάπτυξης. Όταν γίνεται μια αλλαγή, τα στάδια τα οποία επιρεάζονται πρέπει να τροποποιούνται κατάλληλα.
	\item Ενσωμάτωση με εργαλεία ελέγχου.
	\item Reverse-engineering: δυνατότητα δημιουργίας κώδικα από δεδομένα
\end{itemize}

	
