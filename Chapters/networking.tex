	\chapter{Δικτύωση}
		Δικτύωση στα ηλεκτρονικά παιχνίδια έχουμε όταν περισσότεροι από ένας παίχτες σε διαφορετικές πλατφόρμες ή υπολογιστές, μοιράζονται και αλληλεπιδρούν στο ίδιο εικονικό περιβάλλον. Παίχτες σε διάφορα σημεία του πλανήτη θέλουν να μοιραστούν ένα εικονικό περιβάλλον σε πραγματικό χρόνο με σκοπό την συνεργασία ή την αντιπαλότητα. O κόσμος είναι ένα υπερσύνολο του offline κόσμου με επιπλέον στοιχεία κοινωνικοποίησης όπως η επικοινωνία μέσω μηνυμάτων ή φωνής.
		
		\subsection{Αλληλεπίδραση σε εικονικό περιβάλλον}	
		Ένας εικονικός κόσμος, περιλαμβάνει πολλές οντότητες οι οποίες αλληλεπιδρούν μεταξύ τους μέσω των μηχανισμών, νόμων και κανόνων που διέπουν τον κόσμο. Παράδειγμα μηχανισμού είναι η προσομοίωση του φυσικού κόσμου, όπου οι οντότητες ανταποκρίνονται σε νόμους της φυσικής.
		
	    Κατά την ενημέρωση του κόσμου, ο προσομοιωτής χρησιμοποιώντας  τους νόμους, τους κανόνες και τους μηχανισμούς που διέπουν τον κόσμο, παίρνει ως είσοδο τις οντότητες, τα ειδικά βάρη των ιδιοτήτων τους, την απόλυτη θέση τους στο σύστημα συντεταγμένων του κόσμου, και το χρονικό διάστημα της προσομοίωσης και αναλύει την προσομοίωση. Η ανάλυση της προσομοίωσης για να είναι επιτρεπτά ακριβής πρέπει να γίνεται περίπου 80-100 φορές το δευτερόλεπτο \cite{gregory2009game}.
		
		Στο τέλος της προσομοίωσης, η μηχανή γραφικών αποτυπώνει τον κόσμο στις εξόδους με στατικά assets ή με αλγόριθμους παραγωγής δυναμικών assets για την αναπαράσταση του κόσμου.
	
		\subsection{Οι απαιτήσεις του υποσυστήματος δικτύωσης}	
		Με βάση τις απαιτήσεις του υποσυστήματος καταλήγουμε στο διάγραμμα ακολουθίας \ref{fig:Network_Sequence_Diagram}.
		
		\begin{figure}[h]
			\centering
			\includegraphics[width=140mm]{Images/gameloop_network_sequence}
			\caption{Διάγραμμα ακολουθίας του υποσυστήματος δικτύωσης}
			\label{fig:Network_Sequence_Diagram}
		\end{figure}		
		
		Βλέποντας το διάγραμμα καταλαβαίνουμε ότι σε η διαδικασία απόδοσης περιλαμβάνει στατικά στοιχεία και αλγόριθμους, τα οποία μπορούν να φορτώνονται τοπικά στον κάθε υπολογιστή, και δεν είναι απαραίτητα για την επίλυση της προσομοίωσης. Τα απαραίτητα στοιχεία για την προσομοίωση τα οποία πρέπει να μοιράζονται μεταξύ των παιχτών είναι:
		\begin{itemize}
			\item Oι ιδιότητες της οντότητας με βάση τους νόμους του κόσμου. Οι ιδιότητες αυτές δεν ενημερώνονται συχνά.
			\item Οι αλλαγές στο σύστημα συντεταγμένων και οι διάφορες ενέργειες της κατευθυνόμενης οντότητας κατά την πάροδο του χρόνου. Οι αλλαγές τοποθεσίας και ενέργειες γίνονται πολλές φορές ανά δευτερόλεπτο. Η προσομοίωση για να είναι ακριβής πρέπει να ενημερώνεται για τις διάφορες ενέργειες σε πραγματικό χρόνο.
		\end{itemize}
				
		\section{Ανάλυση εννοιών}	
		Η ανάγκη αποστολής πολλών μηνυμάτων ανά δευτερόλεπτο οδηγεί στη χρήση των sockets δικτύου (network sockets). Τα sockets χρησιμοποιώντας ένα \gls{IP} και \gls{port} και επιτρέπουν την αποστολή και παραλαβή μηνυμάτων με βάση κάποιου πρωτοκόλλου.
		
		\subsection{Πρωτόκολλο}
		Τα πιο συχνά χρησιμοποιούμενα πρωτόκολλα για δικτύωση σε παιχνίδια είναι τα παρακάτω \cite{sanja14}:
		\begin{description}	
			\item [TCP] To TCP (transmission control protocol), είναι το πιο συχνά χρησιμοποιημένο πρωτόκολλο. Η διασύνδεση με TCP είναι αξιόπιστη και τα μηνύματα παραλαμβάνονται στη σειρά αποστολής. Η αξιοπιστία όμως έρχεται με ένα μικρό κόστος απόδοσης.
			\item [UDP] Το UDP (user diagram protocol) δεν περιλαμβάνει την αξιοπιστία και την εγγύηση της αλληλουχίας μηνυμάτων. Η απουσία λειτουργιών όμως, το κάνει το πιο γρήγορο σε αποστολή μηνυμάτων πρωτόκολλο.
			
			Η επιλογή πρωτοκόλλου γίνεται ανάλογα με το γενικότερο πλαίσιο και τη συγκεκριμένη χρήση του πακέτου αποστολής. Στην αποστολή της τοποθεσίας, η οποίο γίνεται 20 φορές ανά δευτερόλεπτο \cite{networkthomson}, η αξιοπιστία δεν είναι το βασικότερο, αλλά η απόδοση. Στην αποστολή των στοιχειών του χρήστη κατά την έναρξη, ή ενώς γραπτού μηνύματος πρέπει να είναι αξιόπιστη.
		\end{description}

		\subsection{Αρχιτεκτονική δικτύου}
			Οι αρχιτεκτονικές δικτύου χωρίζονται ανάλογα με το που γίνεται η επίλυση και επεξεργασία της προσομοίωσης.  
		\begin{description}	
			\item [Client-server model] Στο client server model (μοντέλο πελάτη εξυπηρετητή) o οποίο ο client απλά αποδίδει στην οθόνη και το μεγαλύτερο κομμάτι της λογικής και της προσομοίωσης τρέχει στον server. Ο server στέλνει οδηγίες στον client για το τι να αποδόσει και ο client απλά υπακούει.
			\item [Client on top of server model] Ο client είναι και server, δηλαδή οι μηχανές που έχουν τον client έχουν και τον server.
			\item [Peer tο peer] Στο peer to peer μοντέλο, οι μηχανές συμπεριφέρονται μερικώς ως clients και μερικώς ως servers, δηλαδή έχουν και στοιχεία λογικής και επεξεργασίας.
		\end{description}
			Η client-server αρχιτεκτονική εφαρμόζεται σε παιχνίδια τα οποία περιλαμβάνουν πολύ μεγάλους κόσμους και η προσομοίωση περιλαμβάνει αλληλεπιδράσεις μεταξύ πολλών οντοτήτων. Οι προσομοιώσεις αυτές γίνονται σε εξειδικευμένες μηχανές και όχι στον προσωπικό υπολογιστή του κάθε χρήστη γιατί χρειάζονται μεγάλους υπολογιστικούς πόρους. Επίσης ο server μπορεί να λύσει race conditions και να λειτουργήσει ως "διαιτητής" μεταξύ δύο οντοτήτων οι οποίες ζητούν πρόσβαση στο ίδιο σύστημα κατά το ίδιο χρονικό διάστημα. Οι αρχιτεκτονικές στις οποίες ο client περιλαμβάνει στοιχεία επεξεργασίας του κοινόχρηστου κόσμου, είναι πιο δύσκολες στην ανάπτυξη και συντήρηση γιατί δημιουργούνται race conditions στο χρονικό διάστημα αποστολής-παραλαβής μηνυμάτων που προορίζονται σε αλληλοεξαρτώμενες λειτουργίες.
			
		\section{Σχεδίαση του υποσυστήματος}
		Κατά το σχεδιασμό διαδικτυακών παιχνιδιών πολλά μοτίβα και στερεότυπα κώδικα επαναλαμβάνονται χωρίς να συμβάλλουν στην λογική ή μηχανισμούς. Σκοπός της βιβλιοθήκης είναι να μειώσει και να απλοποιήσει τον επαναλαμβανόμενο κώδικα και να προμηθεύσει τον χρήστη με μοτίβα και μοντέλα ούτως ώστε να εστιάσει μόνο στα απαραίτητα.
		
		\subsection{Οντότητες}
		\paragraph{Serialization}
		Σε μια αντικειμενοστραφή γλώσσα χρησιμοποιούνται κλάσεις για να μοντελοποιήσουν το πρόβλημα. Όμως τα αντικείμενα τον κλάσεων δεν μπορούν να σταλούν μέσω socket δικτύου στην αρχική τους μορφή, πρέπει να γίνει μια μετατροπή σε binary για να είναι συνεπής με τη μορφή των δεδομένων που αποστέλλονται μέσω του socket. Κατά την αποστολή έχουμε το serialization τον αντικείμενων, και κατά την παραλαβή το deserialization του πακέτου στο αντικείμενο που αντιπροσωπεύει.
		\subsection{Τύποι μηνυμάτων}
		Οι διάφοροι τύποι μηνυμάτων χρησιμοποιούνται για να ξεχωρίσουν την κατάσταση στην οποία βρίσκεται ο client ή ο server.
			\begin{description}
				\item [Αλλαγή κατάστασης] Τα παρακάτω μηνύματα αποστέλλονται όταν συμβεί αλλαγή κατάστασης του παίχτη στο δίκτυο του παιχνιδιού.
				\begin{description}
					\item [Connecting] Ο παίχτης ξεκίνησε τη διαδικασία σύνδεσης.
					\item [Connected] Ο παίχτης συνδέθηκε.
					\item [Disconnecting] Ο παίχτης ξεκίνησε τη διαδικασία αποσύνδεσης.
					\item [Disconnected] Ο παίχτης αποσυνδέθηκε.
				\end{description}
				\item [Data] Το μήνυμα περιέχει δεδομένα.
				\item [ErrorMessage] Το μήνυμα περιέχει πληροφορίες για σφάλμα.
				\item [WarningMessage] Το μήνυμα περιέχει προειδοποίηση.
				\item [VerboseMessage] To μήνυμα περιέχει γενικές πληροφορίες για την κατάσταση του δικτύου.
				\item [ConnectionApproval] Το μήνυμα σηματοδοτεί την έγκριση στο δίκτυο
				\item [DiscoveryRequest] Σε περιπτώσεις ανιχνεύσης server σε δίκτυο, ο παίχτης στέλνει το αίτημα ανίχνευσης.
				\item [DiscoveryResponse] Σε επιτυχείς ανιχνεύσεις ο server επιστρέφει απάντηση με πληροφορίες. 
			\end{description}
		Οι τύποι μηνυμάτων μπορούν να μοντελοποιηθούν ως ένα enumeration και να προστεθεί η πληροφορία τους στο πρώτο byte του serialized μηνύματος. Το πρώτο byte του παρεληφθέντα μηνύματος περιλαμβάνει τον τύπο του μηνύματος.
		
		\subsection{Διαχειριστής δικτύωσης}
		Ο διαχειριστής δικτύωσης (network manager) είναι υπεύθυνος για την διαχείριση των sockets, την αποστολή και παραλαβή μηνυμάτων, την ενθυλάκωση των λειτουργιών για την διαδικτύωση, είναι ο πυρήνας της υποβιβλιοθήκης και επεκτείνεται από τον client και τον server οι οποίοι προσθέτουν λειτουργίες.
			
		\section{Υλοποίηση}	
		\subsection{Τρόπος χρήσης}
		Η διαδικασία χρήσης περιγράφεται στο διάγραμμα \ref{fig:Network_Usage_Diagram}. Η χρήση πρέπει να είναι εξορθολογιστική και ξεκάθαρη για εύκολη και γρήγορη ανάπτυξη και περιοριστική για αποφυγή σφαλμάτων.
		\begin{figure}
			\centering
			\includegraphics[width=100mm]{Images/network_usage_diagram}
			\caption{Διάγραμμα χρήσης του υποσυστήματος δικτύωσης}
			\label{fig:Network_Usage_Diagram}
		\end{figure}	
					
		\subsection{Αρχιτεκτονική}	
		Η αρχιτεκτονική στηρίζεται στην ανταλλαγή μηνυμάτων-κλάσεων και χωρίζεται στα παρακάτω επίπεδα.
			\begin{description}
				\item [Packages] Χειρίζεται το serialization, αναλύει τα εισερχόμενα μηνύματα και χειρίζεται τις ανακατευθύνσεις του εκτελούμενου κώδικα κατά την παραλαβή μηνυμάτων. 
				\item [Networking] Διαχειρίζεται τις συνδέσεις των χρηστών, τα sockets και ο,τι έχει να κάνει με διασύνδεση.
				\item [Auditing] Αφαιρετικό επίπεδο για logging. Ο χρήστης μπορεί να ενσωματώσει τη δική του λογική παρακολούθησης μηνυμάτων.
				\item [Infrastructure] Το επίπεδο αυτό περιέχει επαναχρησιμοποιήσιμο κώδικα των πακέτων όπως το ConcurrentRepository, για thread-safe αποθήκευση κλειδιών και τιμών, και το ParallelBufferBlock το οποίο εκτελεί κώδικα σε buffer άλλου thread για ασύγχρονη επεξεργασία.
			\end{description}
			
			\subsection{Ανταλλαγή πακέτων}	 
			Το serialization των μηνυμάτων πρέπει να είναι ντετερμινιστικό ώστε να αναγνωρίζεται ο τύπος του μηνύματος και η κλάση την οποία αντιπροσωπεύει και ελαφρύς ώστε να μην επιβαρύνεται το δίκτυο με επιπλέον δεδομένα. Ο ενσωματωμένος serializer χρησιμοποιεί reflection για να αναγνωρίσει τα primitive properties της κλάσης. Για προχωρημένο serialization περίπλοκων τύπων, ο χρήστης έχει τη δυνατότητα να συμπεριλάβει τον δικό του serializer με την υλοποίηση του IPackageSerializer interface και την καταχώρησή του σημειώνοντας την κλάση με το PackageSerializerAttribute και τον τύπο του serializer.
						
			Κατά την προετοιμασία του network manager, o χρήστης καλείται να καταχωρίσει στο container ποιες κλάσεις θα χρησιμοποιηθούν κατά την ανταλλαγή μηνυμάτων. Οι κλάσεις που προορίζονται για serialization σημειώνονται με κάποιο Attribute και η καταχώριση μπορεί να γίνει δυναμικά με reflection.
			Όταν τελειώσει η καταχώριση, το container χρησιμοποιώντας ένα ντετερμινιστικό αλγόριθμο ταξινομώντας με βάση το όνομα της κλάσης στο χώρο ονομάτων κτίζει ένα thread safe repository στο οποίο καταχωρείται ένα μοναδικό byte για αναγνωριστικό του κάθε τύπου και πληροφορίες για την χρήση του τύπου.
			
			Στο τέλος της καταχώρισης και της κατασκευής αλγόριθμου αντιστοίχισης αντικειμένων και λογικής, ο χρήστης μπορεί να καταχωρίσει εκτελέσιμο κώδικα υπό μορφή function o οποίος εκτελείται ασύγχρονα κατά την παραλαβή κάποιου μηνύματος-κλάσης, τύπου μηνύματος, ή συμβάντος συγκεκριμένης σύνδεσης.
			
			Στο \gls{UML} διάγραμμα \ref{network_packages} παρουσιάζεται το υποσύστημα διαχείρισης πακέτων.
			\begin{figure}
				\centering
				\includegraphics[width=165mm]{Images/network_architecture_packages}
				\caption{Υποσύστημα διαχείρισης πακέτων}
				\label{network_packages}
			\end{figure}	
					
			\subsection{Διαχείριση δικτύωσης}	 					
			
			Οι ρυθμίσεις του network manager κτίζονται από αφαιρέσεις. Η χρήση πρωτοκόλλων και τεχνικών γίνεται με άγνοια της υλοποίησης και των λεπτομερειών.
			Η σχεδίαση επιτρέπει την αποστολή μηνύματος ανεξαρτήτου πρωτοκόλλου, τρόπου αποστολής και παραλήπτη. Ο network manager περιλαμβάνει τεχνικές function lifting για αποστολή μηνυμάτων. Ο χρήστης μπορεί να ρυθμίσει διάφορους τρόπους αποστολής ανάλογα με το περιβάλλον χρήσης, και να τους κρύψει πίσω από functions. Η αρχιτεκτονική του networking module παρουσιάζεται στο \gls{UML} διάγραμμα \ref{networking_module}
						
			\begin{figure}
				\centering
				\includegraphics[width=165mm]{Images/network_architecture_networking}
				\caption{Μονάδα διαχείρισης της δικτύωσης}
				\label{networking_module}
			\end{figure}		
	\newpage
	\subsection{Παραδείγματα API}
	Στα παρακάτω παραδείγματα, γίνεται μια σύντομη επισκόπηση του \gls{API}.
	\begin{itemize}
		\item Κλάσεις που προορίζονται για ανταλλαγή μηνυμάτων
	\lstset
	{
		style=sharpc, 
		caption={Attribute πακέτων}
	}
		\begin{lstlisting}
[GinetPackage]
 //optional
[PackageSerializer(typeof(MyCustomSerializer))]
public class MyPackage
{
    public string Message { get; set; }
}          
		\end{lstlisting}
		\item Ρύθμιση και εξατομίκευση server
	\lstset
	{
		style=sharpc, 
		caption={Ρύθμιση server}
	}
		\begin{lstlisting}     
var server = new NetworkServer("MyServer",
builder =>
{
	//via reflection
	builder.RegisterPackages(
		Assembly.Load("Packages.Assembly.Name"));
	//or manually
	builder.RegisterPackage<MyPackage>();
}
cfg =>
{
	cfg.Port = 1234;
	cfg.ConnectionTimeout = 5.0f;
	cfg.MaxConnections = 10;
	//Additional configuration
}
//optional, logging output, default is the one supplied			 
out: new ActionAppender(Console.WriteLine) 
		\end{lstlisting}
		\item Καταγραφή κινήσεων		
	\lstset
	{
		style=sharpc, 
		caption={Καταγραφή κινήσεων}
	}
		\begin{lstlisting}
server.IncomingMessageHandler.LogTraffic();
		\end{lstlisting}		
		\item Απάντηση κατά την παραλαβή μηνύματος
	\lstset
	{
		style=sharpc, 
		caption={Απάντηση κατά την παραλαβή μηνύματος}
	}
		\begin{lstlisting}
server.Broadcast<ChatMessage>((msg, im, om) =>
{
	server.Out.Info($"Received {msg.Message}");
	server.SendToAllExcept(om, im.SenderConnection, NetDeliveryMethod.ReliableOrdered, channel: 0);
}, 
packageTransformer: msg => 
msg.Message += "this is broadcasted");	


server.BroadcastExceptSender<ChatMessage>((sender, msg) =>
{
	server.Out.Info($"Broadcasting {msg.Message}. Received from: {sender}");
});
		\end{lstlisting} 		

		\item Κώδικας ο οποίος εκτελείται κατά την παραλαβή συγκεκριμένου τύπου μηνύματος
	\lstset
	{
		style=sharpc, 
		caption={Χειρισμός εισερχόμενων τύπων-πακέτων}
	}
		\begin{lstlisting}
server.IncomingMessageHandler.OnMessage(
	NetIncomingMessageType.ConnectionApproval, 
	incomingMsg =>
{
	//Configure connection approval
	var parsedMsg = server
		.ReadAs<ConnectionApprovalMessage>(incomingMsg);
	if(parsedMsg.Password ==
		 "my secret and encrypted password")
	{
		incomingMsg.SenderConnection.Approve();
		incomingMsg.SenderConnection.Tag = 
			parsedMsg.Sender;
	}
	else
	{
		incomingMsg.SenderConnection.Deny();
	}
});
		\end{lstlisting}
		\item Εκτελέσιμος κώδικας κατά την παραλαβή συγκεκριμένου μηνύματος-κλάσης
	\lstset
	{
		style=sharpc, 
		caption={Χειρισμός εισερχόμενων μηνυμάτων}
	}
		\begin{lstlisting}
server.IncomingMessageHandler
.OnPackage<MyPackage>((msg, sender) => 
	Console.WriteLine(
	$"{msg.Message} from {sender.SenderConnection}"));
		\end{lstlisting}	
		\item Αποστολή μηνύματος-κλάσης
		\lstset
		{
			style=sharpc, 
			caption={Αποστολή μηνύματος-κλάσης}
		}
		\begin{lstlisting}
server.Send(
	new MyPackage { Message = "Hello" },
(outgoingMessage,peer) => 
	peer.SendMessage(outgoingMessage,
		NetDeliveryMethod.ReliableOrdered);
		\end{lstlisting}		
		\item Lift αποστολής μηνύματος
	\lstset
	{
		style=sharpc, 
		caption={Lift αποστολέα μηνυμάτων}
	}
		\begin{lstlisting}
var packageSender = client.LiftSender((msg, peer) =>
	peer.SendMessage(msg, 
		NetDeliveryMethod.ReliableOrdered));
		
packageSender.Send(new MyPackage { Message = "Hello" });	
		\end{lstlisting}	
		\item Διαγραφή χειριστή
	\lstset
	{
		style=sharpc, 
		caption={Διαγραφή χειριστή}
	}
		\begin{lstlisting}	
var handlerDisposable = server.IncomingMessageHandler
	.OnPackage<MyPackage>((msg, sender) => {}));
		
handlerDisposable.Dispose();	
		\end{lstlisting}		
		\item Η ρύθμιση και το API του client είναι πανομοιότυπο με του server. Στη θέση του \textit{NetworkServer} χρησιμοποιείται ο \textit{NetworkClient} όπου και τα δύο υλοποιούν τον \textit{NetworkManager<TPeer> where TPeer:NetPeer}		
	\end{itemize}
	
	\section{Testing}
	Μια καλή αρχιτεκονική υποστηρίζεται από την ευκολία της δοκιμαστικότητας ανά μονάδα (unit testability). Η βιβλιοθήκη χρησιμοποιεί functional τεχνικές, οι οποίες εύκολα μπορούν να αντικατασταθούν με \gls{mocks}. Ο χρήστης μπορεί να μιμηθεί την παραλαβή ή αποστολή πακέτων και να ενημερώνει την μονάδα επεξεργασίας μηνυμάτων από το τρέχων thread.
	
	\lstset
	{
		style=sharpc, 
		caption={Δοκιμαστικότητα εισερχόμενου πακέτου}
	}
	\begin{lstlisting}
	
[TestMethod]
public void Incoming_Package_GetsHandled()
{
	//Arrange
	var client = new NetworkCliient(
		builder => 
			builder.Register<MyPackage>(),
		cfg=> {});
		
	var mockedHandler = 
		new Mock<Func<MyPackage,NetIncomingMessage>>();
	mockedHandler.Setup( x => 
		x(It.IsAny<MyPackage>()));
	
	client
	.IncomingMessageHandler
	.OnPackage<MyPackage>(mockedHandler);	
	
	//Act	
	client.IncomingMessageHandler.SimulateReceive(
		new MyPackage());	
	client.ProcessMessages().RunSynchronously();
	
	//Assert
	mockedHandler.Verify(x=>x(), Times.Once()));
}
	\end{lstlisting}
		
	Επίσης παρέχεται η δυνατότητα της προσομοίωσης χαμένων πακέτων, για να δοκιμαστεί το πώς η εφαρμογή ανταποκρίνεται σε περιβάλλον κακής διαδικτύωσης και να αναπτυχθούν τεχνικές πρόβλεψης δικτύου (network prediction) για εξομάλυνση της κίνησης.
	