	\begin{Abstract}
		Τα πρώτα ηλεκτρονικά παιχνίδια είχαν γραφτεί εξ'ολοκλήρου σε υλισμικό (hardware). Από τότε, οι κάρτες γραφικών και οι μικροεπεξεργαστές βελτιώθηκαν, δημιουργήθηκαν κονσόλες φτιαγμένες αποκλειστικά για ηλεκτρονικά παιχνίδια, με ειδικά χειριστήρια τα οποία σου προσφέρουν διαφορετικές εμπειρίες.
		Η διαδικασία ανάπτυξης λογισμικού είναι ακριβή και ο σχεδιασμός γίνεται όλο και πιο σύνθετος και περίπλοκος. Τα έργα γίνονται όλο και πιο απαιτητικά και δαπανηρά. Δημιουργήθηκε η ανάγκη για ένα εργαλείο το οποίο να παρέχει ένα ομοιογενές περιβάλλον για την ανάπτυξη σύνθετων έργων. 
		Ένα CASE (Computer Aided Software Engineering) tool είναι ένα λογισμικό-εργαλείο το οποίο απλοποιεί τον κύκλο ανάπτυξης ενός λογισμικού. 
		Στο τομέα του σχεδιασμού παιχνιδιών το πιο διαδεδομένο CASE tool είναι η μηχανή γραφικών. Μια μηχανή γραφικών είναι μια σουίτα από επαναχρησιμοποιήσιμα οπτικά εργαλεία τα οποία βρίσκονται σε ένα ενιαίο περιβάλλον.
		Σκοπός της πτυχιακής είναι να αναγνωριστούν μοτίβα και τεχνικές δημιουργίας παιχνιδιών, ώστε να δημιουργηθεί ένα εργαλείο το οποίο να το προσεγγίζει από υψηλό επίπεδο με αφαιρέσεις για εύκολη μοντελοποίηση και αυτοματοποίηση κατά τη δημιουργία.
	\end{Abstract}
	
	\begin{Acknowledgement}
			Η πτυχιακή μου εργασία είναι αφιερωμένη στους γονείς μου Χαράλαμπο και Παναγιώτα και τον αφελφό μου Κώστα για την στήριξη και υπομονή τους κατά τη διάρκεια των φοιτητικών μου χρόνων. Επίσης ευχαριστώ θερμά τον επιβλέποντα καθηγητή μου, κύριο Νίκο Πεταλίδη, για όλη τη βοήθεια, καθοδήγηση που μου προσέφερε και τη γνώση που μου μετέφερε. Τέλος ευχαριστώ τον συνεργάτη μου Δήμο Παύλου για την εμπειρία που απέκτησα μέσω της επαγγελματικής μας σταδιοδρομίας.
	\end{Acknowledgement}