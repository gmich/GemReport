	\chapter{Game Engineering}
			
	\section{Μηχανές Γραφικών}
	Στο τομέα του σχεδιασμού παιχνιδιών το πιο διαδεδομένο CASE tool είναι η μηχανή γραφικών. Μια μηχανή γραφικών είναι μια σουίτα από επαναχρησιμοποιήσιμα οπτικά εργαλεία τα οποία βρίσκονται σε ένα ενιαίο περιβάλλον.
	Η κεντρική λειτουργικότητα η οποία παρέχεται περιλαμβάνει τη φωτοαπόδοση σε πραγματκό χρόνο (real time rendering) , τη μηχανή φυσικής και εντοπισμό συγκρούσεων (physics and collision detection), το scripting, το animation, την τεχνητή νοημοσύνη (Artificial Intelligence), τη δικτύωση (networking), τον παραλληλισμό ενεργειών (multitasking), την διαχείριση μνήμης και τον γράφο σκηνής (scene graph). Η ανάπτυξη τον παιχνιδιών μέσω μιας μηχανής γραφικών γίνεται εύκολα, γρήγορα και οδηγούμενη απο δεδομένα (data driven) ούτως ώστε οι δημιουργοί παιχνιδιών να μπορούν να ασχολούνται με τις λεπτομέρειες του παιχνιδιού τους.
	Οι μηχανές αναπτύσσονται από ομάδες που απαρτίζονται όχι μόνο από προγραμματιστές, αλλά και απο μαθηματικούς, φυσικούς κλπ. Η κάθε υπο-ομάδα εστιάζει σε ένα συγκεκριμένο κομμάτι, όπως οι φυσικοί με τον εντοπισμό συγκρούσεων, και αρχιτέκτονες λογισμικού σχεδιάζουν το πως τα κομμάτια συνδέονται και αλληλεπιδρούν μεταξύ τους, χωρίς να τους απασχολούν οι λεπτομέριες σχεδίασης του κάθε κομματιού.
	
	\subsection{Ιστορία}
	Η ιστορία των Βιντεοπαιχνιδιών, αρχίζει στα τέλη της δεκαετίας του '40. Προς τα τέλη του '50 και στα μέσα του '60, στην Αμερική, αρχίζουν να μπαίνουν στην καθημερινή μας ζωή, οι υπολογιστές. Για την ακρίβεια, οι κεντρικοί υπολογιστές. Από εκείνη την περίοδο, τα βιντεοπαιχνίδια έκαναν την εμφάνιση τους, στις κονσόλες, στα φλίπερ, στους υπολογιστές, αλλά και στις φορητές κονσόλες. Από τότε η δημιουργία παιχνιδιών έχει γιγαντιωθεί έχοντας ένα τεράστιο κομμάτι της παγκόσμιας οικονομίας.
	Πλέον ο ανταγωνισμός είναι τεράστιος, τα βιντεοπαιχνίδια κυκλοφορούν για διάφορες κονσόλες με  πολύ απαιτητικά γραφικά και με πολύ γρήγορο ρυθμό.
	
	\section{Δομή μιας τυπικής ομάδας ανάπτυξης παιχνιδιών}
	Πριν από την ανάλυση της δομής της μηχανής, θα γίνει ανάλυση της δομής ομάδας η οποία θα την χρησιμοποιεί για να αναπτυχθούν στοχευμένα εργαλεία για το κάθε πρόβλημα της κάθε υπο-ομάδας.
	
	\subsection{Μηχανικοί}	
	Οι μηχανικοί σχεδιάζουν και υλοποιούν το λογισμικό του παιχνιδιού και τα εργαλεία τα οποία χρησιμοποιούνται για την ανάπτυξή του. Οι δύο μεγάλες κατηγορίες μηχανικών είναι οι
	\begin{itemize}
		\item runtime programmers οι οποίοι ασχολούνται με τη μηχανή κται το παιχνίδι 
		\item tool programmers οι οποίοι γράφουν tools τα οποία αυτοματοποιούν και ευκολύνουν την διαδικασία ανάπτυξης.
	\end{itemize}
	Οι μηχανικοί έχουν είτε κάποια ειδικότητα, για παράδειγμα ειδικότητα στη τεχνητή νοημοσύνη, είναι είναι generalists, δηλαδή κατέχουν από όλα τα στοιχεία και μπορούν να λύσουν προβλήματα που κατά τη διάρκεια ανάπτυξης.
	
	\subsection{Artists}
	Οι artists παράγουν όλο το οπτικοακουστικό κομμάτι του παιχνιδιού, το οποίο είναι βασικό κομμάτι για το χαρακτήρα του παιχνιδιού. Χωρίζονται στις εξής κατηγορίες
	
	\begin{itemize}
		\item Concept artists οι οποίοι σχεδιάζουν σκίτσα και πίνακες τα οποία παρέχουν στην ομάδα την εικόνα του τελικού παιχνιδιού. Παρέχουν οπτική καθοδήγηση στην ομάδα καθ' όλη τη διάρκει του κύκλου ανάπτυξης.
		\item 3D Modelers οι οποίοι είναι υπεύθυνοι για την τρισδιάστατη γεωμετρία του εικονικού κόσμου του παιχνιδιού. Απαρτίζονται από τους
		foreground modelers οι οποίοι σχεδιάζουν χαρακτήρες, οχήματα, οπλα και αντικείμενα του τρισδιάστατου κόσμου
		background modelers οι οποίοι σχεδιάζουν την στατικό περιβάλλον πχ κτήρια
		\item Texture artists οι οποίοι σχεδιάζουν τις δισδιάστατες εικόνες που καλύπτουν τα τρισδιάστατα μοντέλα
		\item Lighting artists που ορίζουν τις στατικές και δυναμικές πηγές φωτός και δουλεύουν με το χρώμα, την κατεύθυνση και την κατεύθυνση του φωτός.
		\item Animators οι οποίοι σχεδιάζουν την κίνηση των χαρακτήρων και των αντικειμένων
		\item Motion capture actors οι οποίοι παρέχουν ακατέργαστα δεδομένα κίνησης για να επεξεργαστούν οι animators και να τα ενσωματώσουν στο παιχνίδι.
		\item Sound designers οι οποίοι παράγουν τα εφέ και τη μουσική.
		\item Voice actors τους οποίους η φωνή ηχογραφείται και χρησιμοποιείται για τους χαρακτήρες στο παιχνίδι
	\end{itemize}
	
	\subsection{Game Designers}
	Η δουλειά ενός game designer είναι να σχεδιάσει το διαδραστικό τμήμα του παιχνιδιού, το gameplay. Ασχολούνται με τον σχεδιασμό επιπέδων, την ιστορία τηις αλληλεπιδράσεις μεταξύ των χαρακτήρων στο παιχνίδι με τους στόχους, σκοπούς και κανόνες του παιχνιδιού.
	Σχεδιάζουν το κάθε επίπεδο μονδικά και αποφασίζουν για τη γεωμετρία στο περιβάλλον, πότε και που εμφανίζονται χαρακτήρες και διάφορα αντικείμενα, πως γίνονται οι μεταβάσεις μεταξύ διάφορων σκηνών κλπ.
	
	\subsection{Producers}
	Ο ρόλος του producer διαφέρει από στούντιο σε στούντιο. Η βασική του δουλειά είναι να προγραμματίζει και να δρομολογεί τις διάφορες εργασίες και να λειτουργεί ως ο συνδετικός κρίκος μεταξύ των ατόμων που παίρνουν ηγετικές αποφάσεις και την ομάδα ανάπτυξης. Οι producers είναι χαρακτηριστικό των ΑΑΑ εταιριών, όπου υπάρχουν πολλά τμήματα και πολλοί εργαζόμενοι.
		
	\section {Τι είναι μια μηχανή γραφικών}
	Η πρώτη αναφορά σε μηχανή γραφικών έγινε στα μέσα της δεκαετίας του 90 και αναφερώταν στο δημοφιλές παιχνίδι Doom του οποίου η αρχιτεκτονική διαχώριζε τα βασικά συστήματα του παιχνιδιού, όπως rendering system, collision detection system, audio system, asset system κλπ. Η αξία αυτού του διαχωρισμού εκτιμήθηκε από την κοινότητα όταν οι προγραμματιστές ξεκίνησαν να πουλάνε άδειες για το λογισμικό, επαναχρησιμοποιούσαν εργαλεία προηγούμενων παιχνιδιών με δημιουργία νέων assets. Μικρότερα στούντιο τροποποιούσαν εκδόσεις υπάρχων παιχνιδιών χρησιμοποιωντας το SDK.ό
	Πολλά παιχνίδια γράφτηκαν με σκοπό να επαναχρησιμοποιηθούν κομμάτια κόδικα και modding. Πολλές μηχανές όπως η μηχανή του Quake III γράφτηκαν με τρόπο ώστε να είναι εύκολα προσαρμόσημες χρησιμοποιώντας scripting, με σκοπό την εμπορευματοποίηση μέσω licensing.
	Η διαχωριστική γραμμή μεταξύ του παιχνιδιού και της μηχανής δεν μπορεί να οριστεί με ακρίβεια. Πολλές μηχανές μπορεί να περιέχουν συγκεκριμένα μέρη που αφορούν συγκεκριμένη λειτουργία του παιχνιδιού. Η μεγάλη διαφορά είναι στο data-driven architecture όπου οι κανόνες και τα στοιχεία δεν είναι hard-coded αλλά διαβάζονται από εξωτερικό αρχείο.
	Οι μηχανές έχουν τα όριά τους ανάλογα με τα είδη παιχνιδιού στα οποία η μηχανή εστιάζει, σε ποιες πλατφόρμες, σε τι στιλ γραφικών, σε ποια αρχιτεκτονική της gpu κλπ. 
	
	\section{Γιατί μηχανές γραφικών;}	
	Η αφαίρεση πάντα βοηθούσε τον εγκέφαλο να λειτουργήσει καλύτερα και να κατανοήσει αλληλεπιδράσεις μεταξύ συστημάτων και περίπλοκες έννοιες. Οι μηχανές γραφικών απαλλάζουν τους γραφίστες και τους προγραμματιστές από τις τεχνικές λεπτομέριες, και εστιάζουν στην αισθητική και στο gameplay. Επίσης με την αποσύνδεση των συστημάτων έχουμε πιο προβλέψημη συμπεριφορά, επεκταστημότητα των υποσυστημάτων ως υποσυστήματα και ευκολη δοκιμαστικότητα.
	
	\begin{figure}
		\centering
		\includegraphics[width=160mm]{Images/game_engine_architecture}
		\caption{Game Engine Architecture}
		\label{fig:Game Engine Architecture}
	\end{figure}	