	\chapter{Ο πηρύνας της Μηχανής}
	
	Ένα framework το οποίο επεκτείνεται και διακλαδώνεται σε πολλές υπο-βιβλιοθήκες, στηρίζεται στη θεμελιώση ενός framework core, του πηρύνα της βιβλιοθήκη. Το \gls{API} του πηρύνα πρέπει να είναι εύκολο στην κατανόηση, να αποτελείται από αυτοεπεξηγούμενες αφαιρέσεις, οι οποίες εκθέτουν μόνο τα απολύτος απαραίτητα, ώστε οι υποβιβλιοθήκες που χρησιμοποιούν τον πηρύνα να μην δεσμεύονται σε υλοποιήσεις, να περιορίζει σε συγκεκριμένο pipeline χρήσης για αποφυγή απρόβλεπτων συμπεριφορών και να προσφέρει δυνατότητες επεκτασιμότητας.  \cite{jaroslav08}
		
	\begin{theorem}[First test theorem]
		This is a test theorem.
	\end{theorem}

	\section{Κύκλος ζωής πηρύνα}	
	Κάθε οντότητα από τη στιγμή της δημιουργίας της στο περιβάλλον της μηχανής μέχρι την καταστροφής της και διαγραφή της από την μνήμη, περνά από κάποια προκαθορισμένα στάδια τα οποία εκτελούνται ανάλογα με την κατάσταση του υλικού και του λογισμικού.
	
	\begin{itemize}
		\item Initialization εκτελείται μόνο κατά τη δημιουργία
		\item Game loop εκτελείται συνέχεια σε βρόγχο
		\item Disposing εκτελείται κατά την καταστροφή
	\end{itemize}
	
	\begin{figure}
		\centering
		\includegraphics[width=80mm]{Images/core_lifecycle}
		\caption{Core Lifecycle}
		\label{fig:corelifecycle}
	\end{figure}
		
	\subsection{Game Loop}
	To game loop εγκυάται τη συνεπής ενημέρωση των υποσυστημάτων ανάλογα με το προκαθορισμένο χρονικό βήμα. Χωρίς το ανεξάρτητο σύστημα ενημέρωσης υποσυστημάτων, η ενημέρωση θα γινόταν στον κύκλο εκτέλεσης του επεξεργαστή με αποτέλεσμα την διαφορετική εμπειρία ανά επεξεργαστή και μηχάνημα.
	
	Το κάθε υποσύστημα έχει διαφορετικές απαιτήσεις για τη βέλτιστη λειτουργία. Το collision detection system μπορεί να χρειάζεται να ενημερώνεται εκατό φορές το δευτερόλεπτο, ενώ το σύστημα τεχνητής νοημοσύνης δύο φορές το δευτερόλεπτο. Η πιο συνηθισμένη τεχνική είναι να υπάρχουν δύο κύρια update loops
	\begin{itemize}
	\item το game loop στο οποίο ενημερώνονται όλα τα υποσυστήματα, physics, logic, dynamics κλπ
	\item rendering loop στο οποίο γίνονται οι κλήσεις στην κάρτα γραφικών και τρέχει στα 50-60 \gls{FPS}
	\end{itemize}
	
	Η διαφορά του χρόνου μεταξύ των ενημερώσεων πρέπει να είναι εγγυημένη για παράδειγμα σε μια μηχανή στην οποία το rendering loop τρέχει στα 60 \gls{FPS} υπάρχει η συχνότητα ενημέρωσης
	\begin{equation}
	 F = 1/T =>  F = 1000ms/60 => F = 16ms. 
	\end{equation}Για να μπορεί να εγγυάται το σύστημα αυτή την αναλογία, πρέπει να μετράει τη διαφορά χρόνου μεταξύ κάθε κλήσης, να εκτελεί τον κώδικα στο συγκεκριμένο loop και για τον υπόλοιπο χρόνο το thread να κοιμάται.
	
	Στα συστήματα πραγματικού χρόνου, η έννοια της διάρκειας και του χρόνου πρέπει να χειρίζεται ως μια ξεχωριστή οντότητα. Ο χρόνος πρέπει να είναι ανεξάρτητος από τον πραγματικό χρόνο. Ένα animation το οποίο γίνεται render σε πραγματικό χρόνο, μπορεί να παίξει αντίστροφα ή με διπλάσια ταχύτητα, αν το χρονοδιάγραμμα στο οποίο ανταποκρίνεται, χειρίζεται τον χρόνο διαφορετικά.
	Όλα τα συστήματα ενημερώνονται γραμμικά συναρτήσει του χρόνου του οποίο παρέχεται ως είσοδος.
	
	\subsection{Υποσυστήματα}
	Το κάθε υποσύστημα δημιουργείται ενημερώνεται και καταστρέφεται μέσα στο γενικό κύκλο ζωής του πηρύνα. Το κάθε υποσύστημα έχει εσωτερικά το δικό του κύκλο ζωής και κύκλο ενημέρωσης ο οποίος λειτουργεί μέσα στην έκταση του εξωτερικού κύκλου ζωής.
	
	\subsection{Τεχνικές Ενημέρωσης Υποσυστημάτων}
	Τα υποσυστήματα με κάποιο τρόπο πρέπει να ενημερώνονται και να επικοινωνούν μεταξύ τους ή να στέλνουν μηνύματα όταν συμβεί κάποιο γεγονός. Υπάρχουν διάφορές τεχνικές ενημέρωσης υποσυστημάτων.
	\begin{itemize}
	\item Message Pumps: τα υποσυστήματα στέλνουν μηνύματα σε ένα message bus και τα συστήματα ενημερώνονται όταν εξυπηρετείται το μήνυμα. Παραμένουν άεργα εφόσον δεν υπάρχει μήνυμα προς εξυπηρέτηση.
	\item Call-back driven: τα υποσυστήματα παρέχουν call-back λειτουργικότητα. Δηλαδή τη δυνατότητα να θέσεις τι κώδικας θα εκτελεστεί κατά κάποιο συμβάν. Ένα συμβάν μπορεί να είναι η σύγκρουση μεταξύ δύο αντικειμένων. Ο χρήστης μπορεί να πει ότι όταν συμβεί σύγκρουση μεταξύ του παίχτη και του εχθρού, ο παίχτης θα χάσει ζωή.
	\end{itemize}
	
	\begin{figure}
		\centering
		\includegraphics[width=160mm]{Images/game_coding_complete_chapter7_gameloops}
		\caption{game loops}
		\label{fig:gameloops}
	\end{figure}
		
	\section{ScreenSystem}
	
	\begin{figure}
		\centering
		\includegraphics[width=160mm]{Images/core_screensystem}
		\caption{core screensystem}
		\label{fig:core_screensystem}
	\end{figure}	
	
	\section{Input System}
	\subsection{Human Interface devices}
	Η μηχανή πρέπει να είναι σε θέση να διαβάζει, να επεξεργάζεται και να χρησιμοποιεί συσκευές ανθρώπινης διεπαφής. Η διαδικασία ανάγνωσης χωρίζεται στις παρακάτω κατηγορίες:
	\begin{itemize}
	\item Polling: η κατάσταση κάποιον συσκευών (κυρίως της παλιάς σχολής) διαβάζεται ρωτώντας τη συσκευή για την κατάστασή της περιοδικά. Αυτό καταλήγει σε πλεονασμό γιατί ρωτάει πολλές φορές χωρίς να παίρνει απάντηση και με μικρή καθυστέρηση γιατί η αλλαγή κατάστασης μπορεί να γίνει μεταξύ ερωτήσεων.
	\item Interrupts (διακοπές): Οι συσκευές στέλνουν δεδομένα μόνο όταν αλλάξει η κατάσταση με κάποιο τρόπο. Ο χρήστης μπορεί γράψει κώδικα και να τον εγγράψει με τρόπο ώστε να εκτελείται μόνο όταν συμβεί κάποια αλλαγή κατάστασης.
	\end{itemize}
	
	\subsection{Τύποι Input}
	
	\begin{itemize}	
	\item Digital Buttons: τα ψηφιακά κουμπιά έχουν δύο καταστάσεις: pressed / not pressed
	\item Analog axes and buttons: τα αναλογικά επιστρέφουν εύρος τιμών: το βαθμό της πίεσης της σκανδάλης ή τη θέση του μοχλού στο δισδιάστατο άξονα. 
	\item Relative Axes: οι αναφορικοί άξονες επιστρέφουν τιμές σε σχέση με το τελευταίο σημείο στο οποίο έγινε κάποια αλλαγή πχ το ποντίκι επιστρέφει τη διαφορά θέσης σε σχέση με το τελευταίο σημείο στο οποίο μετακινήθηκε.
	\item Accelerators: Ανιχνεύουν τρισδιάστατες επιταχύνσεις.
	\item Sensor bars: Αισθητήρες όπως οι κάμερες.
	\end{itemize}
	
	\subsection{Απαιτήσεις υποσυστήματος}
	Η οντότητα θέλει να εκτελέσει κώδικα προσανατολισμένο κατά συμβάν εισόδου από συσκευή ανθρώπινης διεπαφής. Ο κώδικας αυτός εκτελείται αντίστοιχα για keyboard-mouse και για gamepad. O χρήστης μπορεί να αλλάξει συσκευή διεπαφής ενώ βρίσκεται σε τρέχον παιχνίδι. Η βιβλιοθήκη δεν πρέπει να στηρίζεται σε κανένα υλικό ή συσκευή. 
	
	\subsection{Observers-listeners}
	Το κάθε υποσύστημα διαχείρισης συσκευών ανθρώπινων διεπαφών αποτελείται από τον διαχειριστή για παράδειγμα τον MouseListener ή Keyboard listener, το οποίο χρησιμοποιεί τις βιβλιοθήκες του τρέχον λειτουργικού για την επιστολή συμβάντων των διεπαφών. Ανάλογα με την πλατφόρμα στην οποία έγινε compile δημιουργούνται οι αντίστοιχοι listeners μέσω του abstract factory.
	Ο χρήστης μπορεί να προσκολήσει κώδικα ο οποίος να εκτελείται ανά συμβαν π.χ. στη μετακίνηση του ποντικιού. Οι listeners ανάλογα με τις συνθήκες επιστολής, όπως το δέλτα του χρόνου επιστολής συμβάντος διπλού click ειναι 200ms, αποστέλλουν τα συμβάντα στoυς καταχωρημένους εκτελεστές. Το κεντρικό σύστημα διαχείρισης εισόδου ενημερώνει όλες τις συνδεδεμένες συσκευές εισόδου. 
	Ο εκτελέσιμος κώδικας μπορεί να γραφτεί μια φορά, ανεξαρτήτως συμβάντος, και να προσκοληθεί στο initialization του lifecycle σε υποσυστήματα διεπαφών.
	
	\begin{figure}[hb]
		\centering
		\includegraphics[width=165mm]{Images/core_input}
		\caption{core input architecture}
		\label{fig:core_input}
	\end{figure}
		
	\lstset
	{
		style=sharpc, 
		caption={Παράδειγμα listener στο initialization}
	}
	\begin{lstlisting}
inputManager.AddListener(new MouseListener
{
	Settings = MouseSettings
		    .DoubleClickDelta(Timespan.FromSeconds(0.2))
			.DragDelta(Timespan.FromSeconds(0.2)),
	Events   = MouseEvents
	        .OnLeftClick(sender,args) => this.Shoot())
	        .OnLeftDoubleClick(sender,args) => this.Roll())      	
});

inputManager.AddListener(new GamepadListener
{
	Settings = GamepadSettings.Default,
	Events   = GamepadEvents
			.OnXButtonPress(sender,args) => this.Shoot())
			.OnYButtonPress(sender,args) => this.Roll())      	
});
\end{lstlisting}
		